\documentclass[12px]{article}
\usepackage[margin=2.5cm]{geometry}
\usepackage[utf8]{inputenc}						% Encodage français
\usepackage[frenchb]{babel}						% Mise en forme française
\usepackage[T1]{fontenc}						% Encodage caractères français
\usepackage{fourier}							% Différents symboles et polices
\usepackage{amsfonts,amsmath,amssymb}			% Symboles maths
\usepackage[francais]{layout}					% Mise en page

\everymath{\displaystyle\everymath{}}			% Toutes les équations en mode \displaystyle

%Quantificateurs
\newcommand\A{\forall}
\newcommand\E{\exists}
%Ensembles
\newcommand\N{\mathbb{N}}
\newcommand\R{\mathbb{R}}
%Pour les intégrales
\newcommand\dx{;\mathrm{dx}}
\newcommand\dt{;\mathrm{dt}}
\newcommand\fdx{f(x)\; \mathrm{dx}}
\newcommand\fdt{f(t)\; \mathrm{dt}}

%Quelques raccourcis utiles (pour les démos du chapitre Intégration)
\newcommand\I{$[a\,;b]$ }

\begin{document}
	\begin{center}
		\fbox{\Huge{L'intégral des démos bac}}
	\end{center}
	
	%====================LIMITES DE SUITES=====================
	%====================1ère Démo du chapitre Limite de suite=====================
	\section{Suite croissante convergente}
	
	\subsection{\'Enoncé}
	Soit ($u_n$) une suite croissante qui converge vers une limite finie $l$.\\
	Alors $\forall n\in\mathbb{N}, \,u_n\leq l$.
	
	\subsection{Démonstration}
	Raisonnons par l'absurde:\\
	Supposons que: $\exists n_0 \in\mathbb{N}, \,u_{n_0}>l$.\\
	Alors, comme la suite $u_n$ est croissante: $\forall n\geq n_{0}, \,u_n\geq u_{n_0}>l$.\\
	L'intervalle ouvert $]l-1; \,u_{n_{0}}[$ contient $l$, mais ne peut contenir les $u_n$ que pour $n<n_0$, donc il ne peut pas contenir tous les termes de la suite à partir d'un certain rang.\\
	Cela contredit le fait que $\lim\limits_{n\rightarrow +\infty}u_n =l$.\\
	Donc $\forall n\in\mathbb{N}, \,u_n\leq l$.
	
	%====================2ème Démo du chapitre Limite de suite=====================
	\section{Suite croissante non majorée}
	
	\subsection{\'Enoncé}
	Une suite croissante non majorée a pour limite $+\infty$.
	
	\subsection{Démonstration}
	Soit ($u_n$) une suite croissante non majorée.\\
	Soit $A$ un réel quelconque.\\
	Comme la suite n'est pas majorée par A, $\exists N\in\mathbb{N}, \,u_{N}>A$\\
	Comme la suite est croissante, $\forall n\geq N, \,u_n\geq u_{N}>A$.\\
	Tous les termes de la suite sont donc dans l'intervalle $]A;\,+\infty[$ à partir d'un certain rang $N$.\\
	Donc $\forall A\in\mathbb{R}, \,\exists N\in\mathbb{N}, \,\forall n\geq N, \,u_n >A \Leftrightarrow \lim\limits_{n\rightarrow +\infty}u_n =+\infty$.
	
	%====================3ème Démo du chapitre Limite de suite=====================
	%\newpage
	\section{Limites des suites géométriques}
	
	\subsection{\'Enoncé}
	$q>1 \Rightarrow \lim\limits_{n\rightarrow +\infty}q^n = +\infty$.
	
	\subsection{Démonstration}
	Soit $q>1$.\\
	Alors $\exists a\in\mathbb{R}_+^*, \,q=1+a$.\\
	\\
	Soit $\forall n\in\mathbb{N}, \,P_n:"q^n\geq 1+na"$.\\
	\textsc{Initialisation:}\\
	Pour $n=0$, $q^0=1$ et $1+0\times a=1$, donc $q^0\geq 1+0\times a$. La récurrence est donc initialisée.\\
	\textsc{Hérédité:}\\
	Supposons que pour un certain $n$ quelconque de $\mathbb{N}$, $P_n$ soit vraie. Montrons que $P_{n+1}$ est vraie.\\
	HR: $q^n \geq 1+na$\\
	Mq: $q^{n+1} \geq 1+a(n+1)$\\
	On a: $q^{n+1}=q\times q^n =q^n(1+a)$\\
	Or, par HR, $q^n \geq 1+na$\\
	Donc $q^{n+1} \geq (1+a)(1+na) = 1+na+a+na^2 = 1+a(n+1)+na^2$\\
	Or, comme $n\geq 0$ et $a^2>0$, $na^2\geq 0$\\
	Donc $q^{n+1}\geq 1+a(n+1)$\\
	La propriété est donc héréditaire.\\
	\textsc{Conclusion:}\\
	$P_0$ est vraie, $\forall n\in\mathbb{N},P_n\Rightarrow P_{n+1}$, donc d'après le principe de récurrence, $\forall n\in\mathbb{N}, q^n\geq 1+na$.\\
	\\
	$a>0 \Rightarrow \lim\limits_{n\rightarrow +\infty}(1+na)= +\infty$.\\
	Donc d'après le théorème de comparaison,$ \lim\limits_{n\rightarrow +\infty}q^n=+\infty$.
	
	%====================EXPONENTIELLE=====================
	%====================1ère Démo du chapitre Exp=====================
	\section{Prérequis (Fonction exponentielle)}
	Soit $f$ une fonction dérivable sur un intervalle $I$.
	Si $\forall x\in I, \,f'(x)=0,$ alors $f$ est constante sur $I$.
	
	\section{1ère Démo (Fonction exponentielle)}
		
	\subsection{\'Enoncé}
	Soit f la fonction dérivable sur $\mathbb{R}$ telle que $f'=f$ et $f(0)=1$.\\
	Alors, $\forall x\in\mathbb{R}, f(x)\times f(-x)=1$ et $f(x)\not= 0$.
		
	\subsection{Démonstration}
	Soit $g$ la fonction définie sur $\mathbb{R}$ par $g(x)=f(x)\times f(-x)$.\\
	Comme $f$ est dérivable sur $\mathbb{R}$, il en est de même de la fonction $u:x\rightarrow f(-x)$ et, pour $x\in\mathbb{R}$, $u'(x)=-f'(x)$.\\
	Donc $g$ est dérivable sur $\mathbb{R}$ et, pour tout $x\in\mathbb{R}$:
	\begin{align*}
		g'(x)=&f'(x)\times u(x) + f(x)\times u'(x)\\
			 =&f'(x)\times f(-x) - f(x)\times f(-x)\\
			 =&f(x)\times f(-x) - f(x)\times f(-x)\quad :\, f'=f\\
			 =&0
	\end{align*}
	Donc g est constante sur $\mathbb{R}$.\\
	Par ailleurs, $g(0)=f(0)f(-0)=(f(0))^2=1$.\\
	Donc $\forall x\in \mathbb{R}, g(x)=1 \Leftrightarrow f(x)\times f(-x)=1$.\\
	\\
	De plus, si $\exists x_0\in\mathbb{R}, f(x_0)=0$, alors $f(x_0)\times f(-x_0)=0$, ce qui contredit le résultat précédent.\\
	Donc $\forall x\in\mathbb{R}, f(x)\not=0$.
	
	
	%====================2ème Démo du chapitre Exp=====================
	\newpage
	\section{Unicité de la fonction exponentielle}
	
	\subsection{\'Enoncé}
	Il existe une unique fonction $f$ définie sur $\mathbb{R}$ telle que $f'=f$ et $f(0)=1$.\\
	Cette fonction s'appelle la \textbf{fonction exponentielle}, et on la note \emph{exp}.
	
	\subsection{Démonstration}
	\textsc{\'Existence:}\\
	L'existence de la fonction exponentielle est admise.\\
	\textsc{Unicité:}\\
	Soit $g$ une autre fonction définie et dérivable sur sur $\mathbb{R}$ telle que $g'=g$ et $g(0)=1$.\\
	D'après la propriété précédente, $\forall x\in\mathbb{R}, f(x)\not=0$, on peut donc définir la fonction:
	\begin{displaymath}
		h(x)=\frac{g(x)}{f(x)}
	\end{displaymath}
	Montrons que $\forall x\in\mathbb{R}, h(x)=1$.\\
	La fonction $h$ est dérivable sur $\mathbb{R}$, donc $\forall x\in\mathbb{R}$:\\
	\begin{align*}
		h'(x)=&\frac{ g'(x)\times f(x) - g(x)\times f'(x) } { (f(x))^2 }\\
			 =&\frac{ g(x)\times f(x) - g(x)\times f(x) } { (f(x))^2 } \quad :f'=f;\; g'=g\\
			 =&0
	\end{align*}
	Donc $h$ est constante sur $\mathbb{R}$.\\
	De plus, $h(0)=\frac{g(0)}{f(0)}=1$.\\
	Donc $\forall x\in\mathbb{R}, \,h(x)=1 \Leftrightarrow \frac{g(x)}{f(x)}=1 \Leftrightarrow g(x)= f(x)$.\\
	Donc $f=g$.
	
	%====================3ème Démo du chapitre Exp=====================
	\section{Limites de la fonction exponentielle}
	
	\subsection{\'Enoncé}
	\begin{enumerate}
		\item $\lim\limits_{x\rightarrow -\infty}e^x = 0^+$
		\item $\lim\limits_{x\rightarrow +\infty}e^x = +\infty$
	\end{enumerate}
	
	\subsection{Démonstration}
	\textsc{Limite en $+\infty$}\\
	Montrons que: $\forall x\in\mathbb{R}^+, e^x >x$.\\
	Soit $\forall x\in\mathbb{R}^+, f(x)=e^x -x$.\\
	$f$ est dérivable sur $[0; +\infty[$ et $\forall x\in\mathbb{R}^+, \,f'(x)=e^x -1$.\\
	$\forall x\geq 0, \,e^x >1 \Rightarrow f'(x)\geq 0$\\
	Donc $f$ est croissante sur $[0;\, +\infty[$ et comme $f(0)= e^0= 1$,\\
	$\forall x\geq 0, \,f(x)\geq 1\Rightarrow f(x)>0 \Leftrightarrow e^x > x$\\
	\\
	Comme $\lim\limits_{x \rightarrow +\infty}x=+\infty$, on a, par comparaison, $\lim\limits_{x\rightarrow +\infty}e^x=+\infty$\\
	\newpage
	\textsc{Limite en $-\infty$}
	\begin{displaymath}
		\forall x\in\mathbb{R}, e^x = \frac{1}{e^{-x}}
	\end{displaymath}
	D'où:\\
	\begin{displaymath}
		\lim\limits_{x\rightarrow +\infty}(-x)=+\infty \;\Rightarrow \lim\limits_{x\rightarrow -\infty}e^{-x}=+\infty \;\Rightarrow \lim\limits_{x\rightarrow -\infty}e^x=0
	\end{displaymath}
	
	%===================INTEGRATION=====================
	%===================1ère Démo du chapitre Intégration====================
	\section{1ère Démo (Intégration)}
	
	\subsection{\'Enoncé}
	Soit $f$ une fonction continue et positive sur un intervalle \I.\\
	Alors la fonction $F$ définie sur \I par :
	\begin{center}
		\begin{displaymath}
			F(x) = \int_{a}^{x}f(t)\,\mathrm{dt}
		\end{displaymath}
	\end{center}
	est dérivable sur \I et $\forall x\in [a;\, b],\, F'(x)=f(x)$.\\
	Plus précisément, $F$ est la primitive de $f$ sur $[a;b]$ qui s'annule en $a$.
	
	\subsection{Démonstration}
	\emph{L'on ne montrera ce théorème que lorsque $f$ est croissante sur \I}.\\
	
	Soit $f$ une fonction continue, positive et croissante sur \I.\\
	Soient $x_0\in\I$ et $h>0$ tel que $x_0+h \in [a;b]$.\\
	
	\emph{Idée: On va encadrer $\frac{F( x_0 +h) - F(x_0) }{h}$ pour calculer sa limite quand $h \rightarrow 0$}\\
		
	On a, d'après la relation de Chasles:\\
	\begin{displaymath}
		F( x_0 +h ) - F(x_0) = \int_{a}^{x_0 +h}\fdt - \int_{a}^{x_0} \fdt = 
		\int_{a}^{ x_0 +h }\dt + \int_{x_0}^{a}\fdt = \int_{x_0}^{x_0 +h} \fdt
	\end{displaymath}
	\\
		
	\emph{Remarque: Comme $f$ est croissante sur \I, le domaine $\mathcal{D}$ est compris entre les rectangles de base $[x_0; x_0 +h]$ et de hauteurs $f(x_0)$ et $f(x_0 +h)$, ce qui va nous permettre d'encadrer $\int_{x_0}^{x_0 +h}\fdt$.}\\
	
	Comme f est croissante sur \I, on a l'encadrement:
	\begin{center}
		\begin{displaymath}
			%Rappel: \leq = <=
			(x_0 +h -x_0 ) \times f(x_0) 
				\leq 
			\int_{x_0}^{x_0 +h}\fdt 
				\leq 
			(x_0 +h -x_0) \times f(x_0 +h)
		\end{displaymath}
	\end{center}
	C'est à dire:
	\begin{center}
		\begin{displaymath}
			%Rappel: \leq = <=
			h \times f(x_0) 
			\leq 
			\int_{x_0}^{x_0 +h}\fdt 
			\leq 
			h \times f(x_0 +h)
		\end{displaymath}
	\end{center}
	D'où, en divisant pas $h>0$:
	\begin{center}
		\begin{displaymath}
			%Rappel: \leq = <=
			f(x_0) \leq \frac{\int_{x_0}^{x_0 +h}\fdt}{h} \leq f(x_0 +h)
		\end{displaymath}
	\end{center}
	Soit encore, puisque $F(x_0 +h) - F(x_0) = \int_{x_0}^{x_0 +h}\fdt$:
	\begin{center}
		\begin{displaymath}
			%Rappel: \leq = <=
			f(x_0) \leq \frac{F(x_0 +h) - F(x_0)}{h} \leq f(x_0 +h)
		\end{displaymath}
	\end{center}
	En procédant de même pour $h<0$, on obtient:
	$f(x_0 +h) \leq \frac{F(x_0 +h) - F(x_0)}{h} \leq f(x_0)$.\\
	Comme f est continue sur \I, on a donc en $x_0,\, \lim\limits_{h \rightarrow 0} f(x_0 +h) = f(x_0)$.\\
	Donc d'après le théorème des gendarmes, 
	$\lim\limits_{h \rightarrow 0}\frac{F(x_0 +h) - F(x_0)}{h} = f(x_0)$.
	Donc F est dérivable sur $x_0$ avec $F'(x_0) = f(x_0)$.\\
	Ceci étant vrai pour tout $x_0$ de \I, F est dérivable sur \I et $F'=f$.
	
	
	%===================2ème Démo du chapitre Intégration====================
	\section{2ème Démo (Intégration)}
	
	\subsection{\'Enoncé}
	Toute fonction continue sur un intervalle admet des primitives sur cet intervalle.
	
	\subsection{Démonstration}
	On se place dans le cas où $f$ est définie sur l'intervalle \textbf{fermé} \I.\\
	On admet que, dans ce cas, $f$ admet un minimum $m$ sur \I.\\
	La fonction $g:x\mapsto f(x)-m$ est alors continue et positive sur \I.\\
	Elle admet donc une primitive $G$ sur \I : $\forall x \in \I, \, G'(x) = f(x) -m$.\\
	Soit $\forall x \in \I, \, F: x \mapsto G(x) + mx$.\\
	Alors $F$ est dérivable sur \I et, pour tout $x \in \I$:
	\begin{center}
		\begin{displaymath}
			F'(x) = G'(x) + m = g(x) - m = f(x) -m +m = f(x)
		\end{displaymath}
	\end{center}
	Ainsi, $f$ admet $F$ pour primitive sur \I.
	
	%===================PROBABILITE=====================
	%===================Démo====================
	\section{Indépendance de deux événements (Probabilités)}
	
	\subsection{Indépendance}
	Deux événements $A$ et $B$ sont indépendants si \fbox{$P(A\cap B) = P(A)\times P(B)$}.\\
	Si $P(A)\not=0$, $A$ et $B$ sont indépendants si, et seulement si \fbox{$P_A(B)=P(B)$}.
	
	\subsection{\'Enoncé}
	Si $A$ et $B$ sont indépendants, alors $\bar{A}$ et $B$ le sont aussi.
	
	\subsection{Démonstration}
	Comme $A$ et $B$ sont indépendants, $P(A\cap B) = P(A)\times P(B)$.\\
	$A$ et $\bar{A}$ forment un système d'événements complet, donc d'après la formule des probabilités totales:\\
	$P(B)= P(A\cap B) + P(\bar{A} \cap B)$\\
	D'où:
	\begin{align*}
	P(\bar{A}\cap B) &= P(B) - P(A\cap B)\\
	&= P(B) - P(A)\times P(B)\\
	&= P(B)\times (1-P(A))\\
	&= P(\bar{A})\times P(B)
	\end{align*}
	Donc $A$ et $B$ sont indépendants.
	
	%===================LOI CONTINUE=====================
	%===================Loi Exponentielle====================
	\section{La loi exponentielle est une loi sans mémoire}
	
	\subsection{\'Enoncé}
	$X\rightsquigarrow\mathcal{E}(\lambda) \Rightarrow \forall (t;\, h)\in\mathbb{R}_+^2,
	\,P_{X\geq t}(X\geq t+h) = P(X\geq h)$
	
	\subsection{Démonstration}
	\begin{align*}
	P_{X\geq t}(X\geq t+h) &= \frac{ P(\,\{X\geq t\} \wedge \{X\geq t+h\,\} ) }{P(X\geq t)}\\
	&= \frac{P(X\geq t+h)}{P(X\geq t)}\\
	&= \frac{ e^{-\lambda (t+h)} }{ e^{-\lambda t} }\\
	&= \frac{ e^{-\lambda t}\times e^{-\lambda h} }{ e^{-\lambda t} }\\
	&= e^{-\lambda h}\\
	&= P(X\geq h)
	\end{align*}
	
	%===================Loi Normale====================
	\section{Unicité de $u_\alpha$}
	
	\subsection{Loi normale centrée réduite}
	\begin{displaymath}
		\forall t\in\mathbb{R},\, \phi (t) = \frac{1}{2\pi}\times e^{ -\frac{t^2}{2} }
	\end{displaymath}
	
	\subsection{\'Enoncé}
	$T\rightsquigarrow\mathcal{N}(0;\, 1)\Rightarrow
	\forall \alpha\in ]0;\, 1[,\, \exists ! u_\alpha\in\mathbb{R}^+,\, P(-u_\alpha\leq T\leq
	u_\alpha) = 1 - \alpha$\\
	
	\subsection{Démonstration}
	Soit $\forall u\in [0; \, +\infty[,\; F(u) = P(-u\leq T\leq u) = 1-\alpha$.\\
	On a: $F(0)=0$ et $\lim\limits_{u\rightarrow +\infty}F(u)=1$ : l'aire sous la cloche vaut 1.\\
	Par ailleurs, par symétrie de $\phi$ par rapport à l'axe des ordonnées, on a:\\
	$F(u) =\, 2\times P(0\leq T\leq u) =\, \int_{0}^{u}\phi (t)\; \mathrm{dt}$\ \  \emph{(avec $\int_{0}^{u}\phi (t)\; \mathrm{dt}$ la primitive de $\phi$ qui s'annule en 0)}.\\
	Donc $F'(u) = 2\times \phi (u) >0$ et par suite, $F$ est strictement croissante sur $[0; +\infty[$.\\
	Donc d'après le théorème des valeurs intermédiaires, comme $(1-\alpha)\in ]0;\; 1[,\; \exists ! u_\alpha\in\mathbb{R},\, F(u_\alpha) = 1-\alpha$,\\
	c'est à dire telle que:
	\begin{displaymath}
		P(-u_\alpha\leq T \leq u_\alpha) = 1-\alpha
	\end{displaymath}
	
	%===================\'ECHANTILLONAGE=====================
	%===================Intervalle de fluctuation avec une loi normale====================
	\newpage
	\section{Intervalle de fluctuation avec une loi normale}
	
	\subsection{\'Enoncé}
	Soient:
	\begin{itemize}
		\item $X_n$ une variable aléatoire suivant la loi binomiale $\mathcal{B}(n;\, p)$
		\item La fréquence $F_n=\frac{X_n}{n}$
		\item $\alpha\in ]0;\, 1[$
		\item $u_\alpha\in\mathbb{R},\, P(-u_\alpha\leq Z\leq u_\alpha)=1-\alpha,\, Z\rightsquigarrow \mathcal{N}(0;\, 1)$
	\end{itemize}
	On note:
	\begin{displaymath}
		I_n = \left[  p-u_\alpha\sqrt{\frac{p(1-p)}{n}} ;\, p+u_\alpha\sqrt{\frac{p(1-p)}{n}} \right]
	\end{displaymath}
	Alors $\lim\limits_{n\rightarrow +\infty}P(F_n\in I_n)=1-\alpha$\\
	Note: $I_n$ s'appelle \textbf{l'intervalle de fluctuation asymptotique} de la fréquence $F_n$ au seuil $1-\alpha $.
	
	\subsection{Démonstration}
	Soit:
	\begin{displaymath}
		Z_n= \frac{X_n-np}{\sqrt{np(1-p)}}
	\end{displaymath}
	D'après le théorème de Moivre-Laplace, quand $n$ devient grand, $Z_n$ suit une loi normale centrée réduite $\mathcal{N}(0;\, 1):$\\
	\begin{displaymath}
		\lim\limits_{n\rightarrow +\infty}P(-u_\alpha\leq Z_n \leq u_\alpha) = 1-\alpha
	\end{displaymath}
	Or:
	\begin{align*}
		-u_\alpha\leq Z_n \leq u_\alpha
		&= -u_\alpha\leq \frac{X_n-np}{\sqrt{np(1-p)}} \leq u_\alpha\\
		&= -u_\alpha\times\sqrt{np(1-p)}\leq\, X_n-np\, \leq u_\alpha\times\sqrt{np(1-p)}\\
		&= np+-u_\alpha\times\sqrt{np(1-p)}\leq\, X_n\, \leq np+u_\alpha\times\sqrt{np(1-p)}\\
		&= p-u_\alpha\times\sqrt{ \frac{p(1-p)}{n} }\leq\, \frac{X_n}{n}\, \leq p+u_\alpha\times\sqrt{ \frac{p(1-p)}{n} }\\
		&= p-u_\alpha\times\sqrt{ \frac{p(1-p)}{n} }\leq\, F_n\, \leq p+u_\alpha\times\sqrt{ \frac{p(1-p)}{n} }
	\end{align*}
	D'où le résultat en passant à la limite.
	
	%===================ESPACE=====================
	%===================Intervalle de fluctuation avec une loi normale====================
	\section{Théorème du toit}
	
	\subsection{\'Enoncé}
	Soient $d_1$ et $d_2$ deux droites parallèles, $\mathcal{P}_1$ et $\mathcal{P}_2$ deux plans distincts tels que $d_1 \subset \mathcal{P}_1$ et $d_2 \subset \mathcal{P}_2$.\\
	Si $\mathcal{P}_1$ et $\mathcal{P}_2$ sont sécants, alors leur droite $\Delta$ d'intersection est parallèle à $d_1$ et $d_2$.
	
	\subsection{\'Enoncé}
	Notons $\vec{u}$ un vecteur directeur de $d_1$ et $d_2$ (qui sont parallèles), et $\vec{w}$ un vecteur directeur de $\Delta$.\\
	Notons $(\vec{u}, \vec{v_1})$ un couple de vecteurs directeurs de $\mathcal{P}_2$.\\\\
	$\Delta \subset \mathcal{P}_1 \Rightarrow \E (x_1; y_1) \in \R^2, \, \vec{w} = x_1\vec{u} + y_1\vec{v_1}$.\\	
	$\Delta \subset \mathcal{P}_2 \Rightarrow \E (x_2; y_2) \in \R^2, \, \vec{w} = x_2\vec{u} + y_2\vec{v_2}$.\\
	
	On a donc $x_1\vec{u} + y_1\vec{v_1} = x_2\vec{u} + y_2\vec{v_2}$, c'est à dire $(x_1 - x_2)\vec{u} = y_2\vec{v_2} - y_1\vec{v_1}$.\\
	
	Si $x_1 \not= x_2$, alors $\vec{u} = \frac{y_2}{x_1 - x_2}\vec{v_1} - \frac{y_2}{x_1 - x_2}\vec{v_2}$ et les vecteurs $\vec{u}$, $\vec{v_1}$ et $\vec{v_2}$ sont donc coplanaires ce qui est impossible car les plans $\mathcal{P}_1$ et $\mathcal{P}_2$ sont sécants.
	Donc $x_1 = x_2$ et $y_1 = 0$ et $y_2 = 0$ car les vecteurs $\vec{v_1}$ et $\vec{v_2}$ ne sont pas colinéaires.\\
	Au final, $\vec{w} = x_1\vec{u}$ et $\Delta$ est parallèle à $d_1$ et $d_2.$
	
	
\end{document}



